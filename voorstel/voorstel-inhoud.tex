%---------- Inleiding ---------------------------------------------------------

% TODO: Is dit voorstel gebaseerd op een paper van Research Methods die je
% vorig jaar hebt ingediend? Heb je daarbij eventueel samengewerkt met een
% andere student?
% Zo ja, haal dan de tekst hieronder uit commentaar en pas aan.

%\paragraph{Opmerking}

% Dit voorstel is gebaseerd op het onderzoeksvoorstel dat werd geschreven in het
% kader van het vak Research Methods dat ik (vorig/dit) academiejaar heb
% uitgewerkt (met medesturent VOORNAAM NAAM als mede-auteur).
% 

\section{Inleiding}%
\label{sec:inleiding}

Mijn graduaatsproef zal gaan over de ontwikkeling van een cross-platform applicatie met gebruik maken van TypeScript/JavaScript. Het wordt een applicatie die gebruikers laat hun gegevens (bv. medische metingen) noteren, bewaren en synchroniseren met de server, waardoor andere clients er toegang toe krijgen. Ik wil dat mijn applicatie volledig offline kan werken, de gegevens zullen in de Local Storage bewaard en op aanvraag gesynchroniseerd.

Mijn werk kan nuttig zijn voor alle ontwikkelaars die zich hebben beperkt tot het publiceren van hun producten alleen als webapplicaties, terwijl standalone applicaties handiger kunnen zijn voor gebruikers. Vanwege de geplande functionaliteiten van mijn applicatie kunnen ziekenhuizen of wijkgezondheidscentra overwegen om deze te implementeren om op afstand door patiënten zelf ingevulde gegevens te kunnen verzamelen.

Mijn onderzoeksvraag:
\textbf{Welke voordelen en moeilijkheden kunnen we krijgen bij het maken van applicaties voor meerdere platformen met gebruik maken van één enkele codebron in JS/TS, het Quasar-framework, Electron en Cordova? } 


%---------- Stand van zaken ---------------------------------------------------

\section{Literatuurstudie}%
\label{sec:literatuurstudie}

In het geval van applicatieontwikkeling zal het gebruik van literatuur vooral plaatsvinden in de online documentatie van de gebruikte frameworks en op web-pagina's zoals StackOverflow en Github op zoek naar oorzaken en oplossingen voor technische problemen.

% Voor literatuurverwijzingen zijn er twee belangrijke commando's:
% \autocite{KEY} => (Auteur, jaartal) Gebruik dit als de naam van de auteur
%   geen onderdeel is van de zin.
% \textcite{KEY} => Auteur (jaartal)  Gebruik dit als de auteursnaam wel een
%   functie heeft in de zin (bv. ``Uit onderzoek door Doll & Hill (1954) bleek
%   ...'')

% Je mag deze sectie nog verder onderverdelen in subsecties als dit de structuur van de tekst kan verduidelijken.

%---------- Methodologie ------------------------------------------------------
\section{Methodologie}%
\label{sec:methodologie}

Het onderzoek zal uitgevoerd worden door het maken van PoC (proef of concept) applicatie en het lezen van de online documentatie van bepaalde frameworks.
Graduaatsproef als applicatieontwikkeling is een taak die niet alleen kennis van programmeertalen vereist, maar ook technische vaardigheden, dus het kan alleen worden gemaakt door mensen met IT-achtergrond.
Het project zal gemaakt worden op een PC laptop met Linux en Windows als virtuele machine (VirtualBox). De code wordt geschreven in de Visual Studio Code, samen met extensies die zijn gewijd aan TypeScript en Vue. Ik ga ook gebruik maken van Android emulator om mobile versie van de applicatie te kunnen testen.


Ik voorzie onderstaande fases van het werk:
\begin{itemize}
  \item Onderzoek van beschikbare cross-platform oplossingen
  \item Een werkende PoC-template applicatie leveren voor alle platformen (Web, Desktop (Windows en Linux), Mobile (Android))
  \item Layout design, Domein- en architectuuranalyse met respect voor OOP
  \item Implementatie van de functionaliteiten
  \item Frontend applicatie koppelen met backend in NestJs
\end{itemize}

Mijn Graduaatsproef zal in het Engels geschreven worden.


%---------- Verwachte resultaten ----------------------------------------------
\section{Verwacht resultaat, conclusie}%
\label{sec:verwachte_resultaten}

% Hier beschrijf je wat je verwacht uit je onderzoek, met de motivatie waarom. Het is \textbf{niet} erg indien uit je onderzoek andere resultaten en conclusies vloeien dan dat je hier beschrijft: het is dan juist interessant om te onderzoeken waarom jouw hypothesen niet overeenkomen met de resultaten.

Als resultaat verwacht ik een stabiel werkende applicatie die identiek eruitziet op alle platformen. De ontwikkeling proces moet ook snel en effectief zijn, met andere woorden wil ik graag nagaan welke moeilijkheden zich tijdens het ontwikkelingsproces kunnen voordoen, waardoor het werk kan worden vertraagd of de kwaliteit kan worden verlaagd. Elk software-ontwikkelingsbedrijf kan veel tijd en dus geld besparen als het product voor alle platforms kan worden getest en gepubliceerd met behulp van één enkele codebron.


\section{Referenties}

\begin{itemize}
  \item \href{https://vuejs.org/guide/}{VueJs documentatie}
  \item \href{https://quasar.dev/introduction-to-quasar}{Introduction to Quasar}
  \item \href{https://cordova.apache.org/docs}{Apache Cordova documentatie}
  \item \href{https://www.electronjs.org/}{ElectronJs documentatie}
  \item \href{https://docs.nestjs.com/}{NestJs documentatie}
\end{itemize}


