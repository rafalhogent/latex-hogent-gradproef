  %  Hier schrijf je de samenvatting van je voorstel, als een doorlopende tekst van één paragraaf. Let op: dit is geen inleiding, maar een samenvattende tekst van heel je voorstel met inleiding (voorstelling, kaderen thema), probleemstelling en centrale onderzoeksvraag, onderzoeksdoelstelling (wat zie je als het concrete resultaat van je bachelorproef?), voorgestelde methodologie, verwachte resultaten en meerwaarde van dit onderzoek (wat heeft de doelgroep aan het resultaat?).
  De graduaatsproef zal proberen antwoord te geven op de vraag of de softwareontwikkelingsmethode op basis van het gebruik van één broncode in JavaScript en TypeScript \autocite{TSDoc} en publicatie op vele platformen effectief betrouwbaar werkende applicatie kan leveren. Die methode kan zeer nuttig zijn voor programmeurs die hun code als standalone applicatie ook kunnen publiceren. Er zal gebruik gemaakt worden van technologieën zoals: Quasar \autocite{QuasarStart}, Vue \autocite{VueDoc}, NestJs \autocite{NestJsDoc}, Electron \autocite{ElectronDoc} en Cordova \autocite{CordovaDoc}.