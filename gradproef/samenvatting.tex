%=============================================================================
% Samenvatting
%=============================================================================

\IfLanguageName{english}{%
\selectlanguage{dutch}
\chapter*{Samenvatting}

% TODO: De "abstract" of samenvatting is een kernachtige
% (~ 1 blz. voor een thesis) synthese van het document.

% Een goede abstract biedt een kernachtig antwoord op volgende vragen:
% 1. Waarover gaat de graduaatsproef? 
% 2. Waarom heb je er over geschreven? 
% 3. Hoe heb je het onderzoek uitgevoerd? 
% 4. Wat waren de resultaten? Wat blijkt uit je onderzoek? 
% 5. Wat betekenen je resultaten? Wat is de relevantie voor het werkveld? 

% Daarom bestaat een abstract uit volgende componenten:
% - inleiding + kaderen thema 
% - probleemstelling 
% - (centrale) onderzoeksvraag 
% - onderzoeksdoelstelling 
% - methodologie 
% - resultaten (beperk tot de belangrijkste, relevant voor de onderzoeksvraag) 
% - conclusies, aanbevelingen, beperkingen 

Cross-platform ontwikkeling biedt vele mogelijkheden voor programmeurs, waarbij één enkele codebase wordt tegelijkertijd omgezet in toepassingen die werken op meerdere verschillende platformen.\newline

De graduaatsproef zal proberen antwoord te geven op de vraag of de softwareontwikkelingsmethode op basis van het gebruik van één broncode en publicatie op vele platformen effectief betrouwbaar werkende applicatie kan leveren. Die methode kan zeer nuttig zijn voor programmeurs die hun code als standalone applicatie ook willen publiceren.\newline

De literatuurstudie helpt om beschikbare oplossingen te vergelijken en om de keuze van de frameworks voor de proof-of-concept-applicatie te rechtvaardigen. \newline

Het project is geschreven in één programmeertaal en geïmplementeerd op verschillende apparaten met één codebase. Er wordt ook een strategie voorgesteld om gebruikersdata te synchroniseren tussen alle frontend-clients.\newline

Ten slotte worden de resultaten geëvalueerd. De moeilijkheden tijdens het ontwikkelingsproces worden benadrukt. De prestaties en stabiliteit van de applicaties worden geanalyseerd, en de grootte van de builds wordt vergeleken.

% LET OP! Een samenvatting is GEEN voorwoord! 


% %---------- Nederlandse samenvatting -----------------------------------------
% TODO: Als je je graduaatsproef in het Engels schrijft,
% moet je eerst een Nederlandse samenvatting invoegen.
% Haal daarvoor onderstaande code uit commentaar.

% Wie zijn/haar graduaatsproef in het Nederlands schrijft,
% kan dit negeren, de inhoud wordt niet in het document ingevoegd.

\selectlanguage{english}
}{}

%%---------- Samenvatting -----------------------------------------------------
% De samenvatting in de hoofdtaal van het document

\chapter*{\IfLanguageName{dutch}{Samenvatting}{Abstract}}

Cross-platform development brings significant opportunities for developers and software houses, simultaneously transforming a single codebase into applications that function across several different platforms. They can accelerate the software production process by enabling developers to create apps without writing separate code for every operating system, therefore reducing costs and delivery time.\newline


The thesis tries to provide the answer to the following question: What benefits and difficulties can be expected when making applications for multiple platforms using a single code source?\newline


The literature was subjected to a detailed analysis, mostly based on online publications, to compare available cross-platform technologies and justify the selection of the frameworks for the proof-of-concept example application. Project was written with one programming language and deployed to different devices using one single codebase. This thesis presents also a strategy to synchronize user data between all the front-end clients.\newline


Finally, the results are evaluated. The difficulties during development process are pointed out. Performance, stability of the applications are analysed, builds sizes are compared.
