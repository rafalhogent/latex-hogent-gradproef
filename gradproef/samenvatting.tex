%=============================================================================
% Samenvatting
%=============================================================================


\IfLanguageName{english}{%
\selectlanguage{dutch}
\chapter*{Samenvatting}

TODO: De "abstract" of samenvatting is een kernachtige (~ 1 blz. voor een thesis) synthese van het document.

Een goede abstract biedt een kernachtig antwoord op volgende vragen:

1. Waarover gaat de graduaatsproef? \newline
2. Waarom heb je er over geschreven? \newline
3. Hoe heb je het onderzoek uitgevoerd? \newline
4. Wat waren de resultaten? Wat blijkt uit je onderzoek? \newline
5. Wat betekenen je resultaten? Wat is de relevantie voor het werkveld? \newline

Daarom bestaat een abstract uit volgende componenten:

- inleiding + kaderen thema \newline
- probleemstelling \newline
- (centrale) onderzoeksvraag \newline
- onderzoeksdoelstelling \newline
- methodologie \newline
- resultaten (beperk tot de belangrijkste, relevant voor de onderzoeksvraag) \newline
- conclusies, aanbevelingen, beperkingen \newline


LET OP! Een samenvatting is GEEN voorwoord! 

%---------- Nederlandse samenvatting -----------------------------------------

TODO: Als je je graduaatsproef in het Engels schrijft, moet je eerst een
Nederlandse samenvatting invoegen. Haal daarvoor onderstaande code uit
commentaar.

Wie zijn/haar graduaatsproef in het Nederlands schrijft, kan dit negeren, de inhoud
wordt niet in het document ingevoegd.

\selectlanguage{english}
}{}

%%---------- Samenvatting -----------------------------------------------------
% De samenvatting in de hoofdtaal van het document

\chapter*{\IfLanguageName{dutch}{Samenvatting}{Abstract}}

