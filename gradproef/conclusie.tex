%%=============================================================================
%% Conclusie
%%=============================================================================

\chapter{\IfLanguageName{dutch}{Conclusie}{Conclusion}}%
\label{ch:conclusie}

% TODO: Trek een duidelijke conclusie, in de vorm van een antwoord op de
% onderzoeksvra(a)g(en). Wat was jouw bijdrage aan het onderzoeksdomein en
% hoe biedt dit meerwaarde aan het vakgebied/doelgroep? 
% Reflecteer kritisch over het resultaat. In Engelse teksten wordt deze sectie
% 'Discussion' genoemd. Had je deze uitkomst verwacht? Zijn er zaken die nog
% niet duidelijk zijn?
% Heeft het onderzoek geleid tot nieuwe vragen die uitnodigen tot verder 
% onderzoek?
% What benefits and difficulties can be expect when making applications for multiple platforms using a single code source?

This thesis has presented a template for design and implementation of a web-technology based cross-platform application development. The quality, performence of the delivered proof-of-concept application is consistent with expectations. 
Despite some difficulties, writing the code and making builds for different operating systems was not a difficult obstacle, instead it delivered interesting experience and opportunity to know the technology better.
\newline
The biggest difficulties include:
\begin{itemize}
    \item maintaining consistent UIs across all platforms
    \item potential conflicts when building dynamicly developing platforms using various technologies
    \item potential limited performance and difficult optimization comparing to native languages
\end{itemize}

The most significant benefits:
\begin{itemize}
    \item accelerating the development process
    \item reducing costs
    \item facilitating the harmonization of UI
    \item faster implementation of changes on all platforms simultaneously
    \item centralizing codebase
\end{itemize}

Future research should explore:
\begin{itemize}
    \item applications performance tests under heavy load using huge amount of data to determine the limits
    \item code sharing between clients and server apps since they use the same language
    \item making builds also for Apple devices: iOS, macOS
    \item trying out other mentioned in state-of-the-art frameworks to compare with the delivered proof-of-concept app
\end{itemize}



